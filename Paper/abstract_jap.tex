\newpage
\thispagestyle{empty}
\begin{center}
	\mbox{\LARGE{\bf{Crowdsourced Manufacturing環境下における}}} \\
	\mbox{\LARGE{\bf{企業間の設備共有手法に関する一提案}}} \\
	\vspace*{2mm}
	\mbox{\Large{\bf{原田佳明}}}\\
	\vspace*{7mm}
	{\LARGE\bf 要旨}
\end{center}\par
顧客ニーズへの対応と生産性の両方を満たすマス・カスタマイゼーションと呼ばれる生産方式が注目を集めている.マス・カスタマイゼーション実現にはIoT技術が必要とされている.IoTを活用し工場やその機器をインターネット上で繋ぐことで製造業の生産性を向上させる必要があり,そのための枠組み作りが世界各国で行われている.また工場内だけでなく,企業を超えた繋がりを利用したCrowdsourced Manufacturingと呼ばれる共生型モノ作りの概念が注目を集めている.Crowdsourced Manufacturing上では各企業や工場をICT環境で繋ぐことにより,設備・材料・労働力・工法を融通することが可能となる.本研究ではCrowdsourced Manufacturing環境下における設備共有による処理工程の融通に着目し,その共有手法の提案することを目的とする.具体的には設備情報の共有を行うことで,企業間で処理工程を委託することを可能にする.オーダが不確定である状況を想定した設備共有手法,オーダが確定している状況を想定した設備共有手法の2つの手法を提案し,そのシミュレーションモデルを構築した.\par
本研究では計算機実験により以下の結果を得た.
\begin{itemize}
	\item オーダが不確定である状況,オーダが確定している状況の両方の場合において設備共有の有効性が確認でき,納期遅れの改善,稼働率の向上が確認できた.
	\item 特にオーダ不確定モデルにおいて納期遅れ改善,稼働率の向上の効果が大きかったが,自社の処理を犠牲にして他企業の処理を行っている場合があると考察された.
	\item オーダ確定している状況においては,不確定オーダモデルほど納期遅れ,稼働率において大きな改善は得られなかったが自社の工程の処理を優先して行えていることが考察された.
\end{itemize}