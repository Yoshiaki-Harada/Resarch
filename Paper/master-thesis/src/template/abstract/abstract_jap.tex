\newpage
\thispagestyle{empty}
\begin{center}
	\mbox{\LARGE{\bf{クラウドソースドマニュファクチャリングに対する}}} \\
	\mbox{\LARGE{\bf{組合せダブルオークションに基づく}}} \\
	\mbox{\LARGE{\bf{リソース配分手法の一提案}}} \\
	\vspace*{2mm}
	\mbox{\Large{\bf{原田佳明}}}\\
	\vspace*{7mm}
	{\LARGE\bf 要旨}
\end{center}\par
従来のモノづくり企業は垂直型経営が主流であった.しかしそれでは近年の製品ライフサ
イクルの短縮化や需要変動に対応できないという課題点が浮かび上がってきた.その課題
を解決する為に日本においてはシェアリング・エコノミーの考え方に基づいたモノづくりの分散化,製
造リソースの共有に関する議論が盛んに行われている.そのような中で,共生型モノづくりのコンセプトであるクラウドソースドマニュファク
チャリングが提案され,注目を集めている.クラウドソースドマニュファクチャリングとは個々の企業が持つリソース情報を共有しその相互活用を行う生
産形態である.この生産携帯の実現には独立した企業が参加する状況下においても成り立つ企業間のリソース配分の仕組
みが必要であるとされている.\\
そこで本研究では,特に買い手・売り手の双方が入札を行える組合せダブルオークション
に基づくに着目し,オークションにおいて重要とされるパレート効率性を満たす
手法と耐戦略性を満たす手法の2つの手法を提案した.そして計算機実験を行うことで
それぞれの手法の特性解析を行うとともに,現実を想定したケーススタディを行うこと
で以下の結論を得た.
\begin{itemize}
\item 手法I:パレート効率性を満たす手法においては, 各企業の総利益が高いパレート
  効率な配分を実現することができたが,参加者企業が評価値を偽って申告することを防ぐ耐戦略性を満たせな
  いことが確認できた.
\item 手法II:耐戦略性を満たす手法においては,定式化,実験両方において耐戦略性を満たす
ことが確認できた.また手法Iより総利益は低くなってしまったが
企業数が増加することほどパレート効率な状態に近づくことが確認できた.
\item 手法Iは虚偽申告により,正しくパレート効率な配分を実現できない可能性がある.
  しかし虚偽申告が行わなければパレート効率な状態を導くことができる.
  従って元々中長期的に信頼関係が築けている企業群によるクラウドソースドマニュファクチャリング
  に有効である.手法IIは耐戦略性を満たせるが,参加企業が少ないと総利益が小さく
  なってしまう.従ってより独立した企業が数多く集まる企業群によるクラウドソー
  スドマニュファクチャリングに有効である.
\end{itemize}