\newpage
\thispagestyle{empty}
\begin{center}
\mbox{\LARGE{\bf{A Proposal on Resource Allocation Method }}}
\mbox{\LARGE{\bf{based on Combinatorial Double Auction}}}
\mbox{\LARGE{\bf{for Crowdsourced Manufacturing }}}

\vspace*{2mm}
\mbox{\Large{\bf{Harada Yoshiaki}}}

\vspace*{7mm}
{\LARGE\bf Abstract}
\end{center}
Mainly, traditional manufacturing companies managed  with vertical
integration. However, the problem has emerged int this mangement style that the
product life cycle in recent years cannot be shortened and the demand cannot be
changed. In order to solve the problem, discussions on the decentralization of
manufacturing and the sharing of manufacturing resources based on the concept of
sharing economy are being actively conducted in Japan. Under such circumstances,
Crowdsourced Manufacturing, a concept of symbiotic manufacturing, has been
proposed and attracted attention. Crowdsourced manufacturing is a production
format in which resource information owned by individual companies is shared and
mutually utilized. In order to realize this production, it is necessary to have
a resource allocation mechanism between companies that work well even
when independent companies participate. \\
In this study, we focused on the combinatorial double auction in which both buyers and sellers can bid, and proposed two methods, one that satisfies Pareto efficiency, and the other that satisfies Stragegy-proofeness . Through computer experiments, the characteristics of each method were analyzed and the methods were compared. The following conclusions were obtained.
\begin{itemize}
\item In the method that satisfies Pareto efficiency, it was possible to realize
  Pareto efficient allocation and high total profit.  However, it was confirmed
  that the company did not have the Strategy-proofness that prevent the
  participant companies bid false value.
\item  In the method that satisfies the Strategy-Proofness, it was confirmed
  that the Strategy-Proofness was satisfied in both the formulation and the
  experiment. Total profirt in Method II lower than Method I, but it was confirmed that as the number of companies increased, this
  total profit was closer to Pareto efficiency.
\item  Since Method I cannot satisfy the Strategy-proofness, it was confirmed
  that Pareto efficient allocation could not be realized correctly due to false
  reporting, and in many cases, the loss was lower than the total profit of Method II.
Therefore, it was confirmed that it is better to use Method I when truthful bids
of participating companies can be expected, and to use Method II in other cases.
\end{itemize}